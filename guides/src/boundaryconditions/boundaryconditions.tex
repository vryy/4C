%%%%%%%%%%%%%%%%%%%%%%%%%%%%%%%%%%%%%%%%%%%%%%%%%%%%%%%%%%%%%%%%%%%%%%
%%%%%%%%%%%%%%%%%%%%%%%%%%%%%%%%%%%%%%%%%%%%%%%%%%%%%%%%%%%%%%%%%%%%%%
\chapter{Boundary conditions}



%%%%%%%%%%%%%%%%%%%%%%%%%%%%%%%%%%%%%%%%%%%%%%%%%%%%%%%%%%%%%%%%%%%%%%
\section{Implementation of periodic boundary conditions}

The periodic boundary conditions which are implemented are suitable only for
rectangular boxes with surfaces oriented parallel to the coordinate planes.

Each pair of boundary conditions defines master and corresponding slave
nodes. The slave nodes get a pointer to the degrees of freedom of their
masternode. This means, that the size of the dofset is reduced in comparison
to a calculation without periodic boundary nodes!

In addition to that, master and slave nodes have to be owned by the same proc
and are only allowed to be ghosted in pairs to be able to assign the degrees
of freedom.

%%%%%%%%%%%%%%%%%%%%%%%%%%%%%%%%%%%%%%%%%%%%%%%%%%%%%%%%%%%%%%%%%%%%%%
\subsubsection{Construction of the master slave matching}
The master slave matching is constructed using a parallel octree search. For
this purpose, we search for each master node the closest slave node on all
processors. For this purpose, one of the slaves coordinates is modified
in such a way that the new coordinate is in the master plane.

%%%%%%%%%%%%%%%%%%%%%%%%%%%%%%%%%%%%%%%%%%%%%%%%%%%%%%%%%%%%%%%%%%%%%%
\subsubsection{Redistribution of nodes}
The nodes are redistributed among the processors. Each couple is owned by a
unique processor and if a slave is ghosted on a processor, the corresponding
master is ghosted there, too.

%%%%%%%%%%%%%%%%%%%%%%%%%%%%%%%%%%%%%%%%%%%%%%%%%%%%%%%%%%%%%%%%%%%%%%
\subsubsection{Renumbering the degrees of freedom}
All slave nodes do not get their own degrees of freedom. Instead, their index
for the access of the degrees of freedom from nodes is set to the index of
their corresponding masternode. (This is what makes it necessary to ghost the
master if you ghost the slave)

%%%%%%%%%%%%%%%%%%%%%%%%%%%%%%%%%%%%%%%%%%%%%%%%%%%%%%%%%%%%%%%%%%%%%%
\subsubsection{Periodic boundary conditions and IO}
The assignement of periodic boundary conditions changes the numbering of the
degrees of freedom. For this reason, the conditions have to be written to the
output to be able to reconstruct the dof pattern during the setup of the
discretisation in the postprocessing filters.

%%%%%%%%%%%%%%%%%%%%%%%%%%%%%%%%%%%%%%%%%%%%%%%%%%%%%%%%%%%%%%%%%%%%%%
\subsubsection{Periodic boundary combined with additional 
  (Dirichlet) conditions}
For periodic boundary conditions several nodes correspond to the same set of
degrees of freedom. If you want to assign an additional condition to such a
node make sure that you assign the same condition to all corresponding nodes.

This necessity is caused by the hirarchical way in which conditions are
assigned to the dofs. Each node belonging to this condition sets its condition
onto the degrees of freedom, overriding previously set conditions.
So, if you want to make sure that this condition is set, you have to assign it
to all coressponding nodes or the result depends on the order in which the
nodes in the condition are processed.

Another thing of practical interest might be that the assignement of periodic
boundary conditions may lead to purely Dirichlet constrained problems
(examples channel flow or a Couette flow). 
