%%
%%############################
\chapter{Welcome to \ccarat{}}

%%
%%============================
\section{History}

\ccarat{} is a research code. It was started by Michael Gee in 2000
and has since grown in several directions. The original aim was to
provide a framework for parallel multifield finite element simulations.
That is the fortran code that was used back then was both serial and
a pain to extend. However, the aim was moderate parallelization, not
massive parallel execution with hundreds of processes. That is why
there is a strong emphasis on data structures in \ccarat{} but still
a lot of data redundancies. The original \ccarat{} is a large C project
that uses external linear solver packages (both parallel and seriel.)

In 2006 it was felt that \ccarat{} needs a major face lift. And again
it is due to Michael Gee to introduce the trilinos libraries into
\ccarat{} and redesign the whole package on top of that.

The original code, however, is still in use and supported. So there
are now two major parts to \ccarat{} that have many ideas in common,
but are based on independent designs. 

%%
%%============================
\section{Overview}

The following Guide to \ccarat{} deals with

\begin{description}
\item[Part I --- Getting started] First steps towards compiling \ccarat{}; \textsc{Hop
  \& Malt} cluster; Tutorials;
  Development tools 
\item[Part II --- Reference guide] Input parameter with short description
\item[Part III --- Development guide] Theoretical background and detailed
  description   on elements, data types, ...
\end{description}