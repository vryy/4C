%%
%%############################
\chapter{Welcome to \baci{}}

%%
%%============================
\section{History}

\baci{} is a C++ research code that has its roots in the by now abandoned
C code \ccarat{}. The base of \ccarat{} was a fortran code, but I will not go
into than. \ccarat{} was started by Michael Gee in 2000 and has since grown in
several directions. The aim back then was to provide a framework for parallel
multifield finite element simulations. That is the fortran code that was used
back then was both serial and a pain to extend. However, the aim was moderate
parallelization, not massive parallel execution with hundreds of
processes. That is why there is a strong emphasis on data structures in
\ccarat{} but still a lot of data redundancies. The original \ccarat{} is a
large C project that uses external linear solver packages (both parallel and
seriel.)

In 2006 it was felt that \ccarat{} needs a major face lift. And again
it is due to Michael Gee to introduce the trilinos libraries into
\ccarat{} and redesign the whole package on top of that. This is how \baci{}
emerged. The \ccarat{} code was replaced step by step (even if some of these
steps were XXL sized) until close to no original code remained.

The original \ccarat{} code is of course still running and usable. But since
nobody wants to touch it any more --- most developers are even afraid to look
at it ---, the dark and mysterious knowledge of its inner workings is about to
fade away.

%%
%%============================
\section{Overview}

The following Guide to \ccarat{} deals with

\begin{description}
\item[Part I --- Getting started] First steps towards compiling \ccarat{}; \textsc{Hop
  \& Malt} cluster; Tutorials;
  Development tools 
\item[Part II --- Reference guide] Input parameter with short description
\item[Part III --- Development guide] Theoretical background and detailed
  description   on elements, data types, ...
\end{description}
