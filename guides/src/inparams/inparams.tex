%%
%%############################
% font
%\usepackage{helvet}
%\renewcommand{\familydefault}{\sfdefault}
% sectioning
%\makeatletter

\newcommand{\subpart}[1]{%
  \vspace{2\parskip}%
  \textbf{#1}%
  \hspace*{1em}%
}
\newcommand{\subsubpart}[1]{%
  \vspace{2\parskip}%
  \textit{#1}%
  \hspace*{1em}%
}
\newcommand{\subitempart}[2][$\bullet$]{%
  \vspace{2\parskip}%
  #1\hspace{1.em}\textbf{#2}\hspace{1.em}%
}
\newcommand{\subsubitempart}[2][--]{%
  \vspace{2\parskip}%
  #1\hspace{1.em}\textit{#2}\hspace{1.em}%
}
\newcommand{\note}[2][]{%
  \textit{#2}\hspace*{1em}
}
\makeatletter
\newcommand\subxsection{\@startsection{subsection}{2}{\z@}%
                                     {-3.25ex\@plus -1ex \@minus -.2ex}%
                                     {1.5ex \@plus .2ex}%
                                     {\normalfont\large\bfseries}}
\makeatother
%\makeatother
% miscellaneous
\newenvironment{inframe}{%
\noindent
\begin{lrbox}{\setbox1}
\begin{minipage}{\linewidth}%
}{%
\end{minipage}
\end{lrbox}
\framebox{\usebox{\box1}}
}
% foot note with choice of symbol 
\long\def\symbolfootnote[#1]#2{\begingroup%
\def\thefootnote{\fnsymbol{footnote}}\footnote[#1]{#2}\endgroup}
\long\def\symbolfootnotetext[#1]#2{\begingroup%
\def\thefootnote{\fnsymbol{footnote}}\footnotetext[#1]{#2}\endgroup}
% convenience
\newcommand{\ger}[1]{\emph{#1}}
\newcommand{\lat}[1]{#1}
\newcommand{\ie}{i.e.\@}
\newcommand{\cf}{cf.\@}
\newcommand{\eg}{e.g.\@}
\newcommand{\intro}[1]{\emph{#1}}
\newcommand{\fail}[1]{\emph{#1}}
%\newcommand{\tc}[1]{\ovalfbox{\textsf{#1}}}
\newcommand{\tc}[1]{%
\ifmmode
  \mathchoice{
    \ovalfbox{\textsf{\text{$#1$}}}
  }{
    \ovalfbox{\textsf{\text{$#1$}}}
  }{
    \ovalfbox{\textsf{\text{$\scriptstyle #1$}}}
  }{
    \ovalfbox{\textsf{\text{$\scriptscriptstyle #1$}}}
  }
\else
\ovalfbox{\textsf{#1}}%
\fi}
\newcommand{\name}[1]{#1}
% rotate
\newcommand{\textrotl}[1]{%
\setbox2=\hbox{#1}
\rotl2
}
% source code style
\newcommand{\cod}[1]{\texttt{\textup{#1}}}
\newenvironment{codenv}{\begin{quote} \tt}{\end{quote}}
\newcommand{\cor}{$|$}
\newcommand{\cgb}{$($}
\newcommand{\cge}{$)$}
\newcommand{\ccb}{$[$}
\newcommand{\cce}{$]$}
\newcommand{\cnl}{$\setminus$}
\newcommand{\chs}{\hspace{3em}}


% bold Figure 7.5 at figures
\makeatletter
\long\def\@makecaption#1#2{%
  \vskip-0.65\abovecaptionskip
  \sbox\@tempboxa{\textbf{#1:} #2}%
  \ifdim \wd\@tempboxa >\hsize
    \textbf{#1:} #2\par
  \else
    \global \@minipagefalse
    \hb@xt@\hsize{\hfil\box\@tempboxa\hfil}%
  \fi
  \vskip\belowcaptionskip}
\makeatother
%%% Local Variables: 
%%% mode: latex
%%% TeX-master: "inventory"
%%% End: 

%\input{underscore}


%%
%%############################
\chapter{Input parameters}

%%
%%============================
\section{Overview}

The total input of \ccarat{} is read from formated input files that contain
all data to define a problem. The file is divided in single blocks
by horizontal lines which end with a key word specifying the following
content. There must not be spaces in front of a keyword. An example
is the \kw{PROBLEM TYP} introductory line 
\begin{verbatim}
-------------------------------------------------------PROBLEM TYP
\end{verbatim}
Plenty of keyword lines and blocks refere to special applications
of the program (such as \kw{OPTIMIZATION}, \kw{ALE DYNAMIC} etc.). These often
can be omitted when not required, which shortens the file and eases
its use. Other blocks however are of general character and must not
be left out in order to obtain a valid input file.

The order of the blocks can be chosen arbitrary while a convention
certainly increases comfort. The single blocks are explained in more
detail subsequently. Emphasis is here put on those blocks which are
of general meaning and are used within all kinds of different problem
types. Specific problem dependent blocks such as \kw{ALE SOLVER}, \kw{FLUID
DYNAMIC} or \kw{OPTIMISATION} can be added to this manually as required. 

%%
%%============================
\section{Comments}
Comments start with $\backslash\backslash$. These can appear in the first
rows or somewhere else. Everything on their right side is a comment and
neglected. 

%%
%%============================
\section{List of input parameters}

%%
%%----------------------------
\subsection{Title}
\begin{verbatim}
-------------------------------------------------------------TITLE
\end{verbatim}

---TITLE 

This block contains the expressive and individual title of your input
problem. 

read in: inpctrhead() in input\_full/input\_ctr\_head.c 

used for: writing it to output file 

written by: hand 

remarks: 

\begin{itemize}
\item The title can be used to set special treatments for selected examples
(which may be advisable for testing only). 
\item The title may be empty or contain up to five lines. 
\item The TITLE head line must not be omitted.
\end{itemize}

%%
%%----------------------------

% ---PROBLEM SIZE

% This block contains the overal size of your input problem. The elements,
% nodes and materials of all involved fields are added here. 

% read in: inpctrprob() in input\_full/input\_control\_global.c

% used for: filling the respective genprob values 

% written by: preprocessor (GiD) 

% contains: 

% \begin{lyxlist}{00.00.0000}
% \item [{ELEMENTS}] total number of elements which are listed in the input
% file 
% \item [{NODES}] total number of FE-nodes which are listed in the input
% file 
% \item [{DIM}] number of spatial dimensions the problem uses, 2 or 3 
% \item [{MATERIALS}] number of materials defined subsequently 
% \item [{NUMDF}] maximal number of degrees of freedom to one FE node 
% \end{lyxlist}
% remarks: 

% \begin{itemize}
% \item There must be at least one material. 
% \item The PROBLEM SIZE block must not be omitted.
% \end{itemize}
% ---PROBLEM TYP 

% This block contains the general information about the type of problem
% to be solved and influences the control routine of the problem. 

% read in: inpctrprob() in input\_full/input\_control\_global.c

% used for: filling the respective genprob values 

% written by: preprocessor (GiD) but check it manually! 

% contains: 

% \begin{lyxlist}{00.00.0000}
% \item [{PROBLEMTYP}] can be Ale, Fluid, Fluid\_Structure\_Interaction,
% Optimisation, Structure, Structure\_Structure\_Interaction 
% \item [{TIMETYP}] specifies Static or Dynamic problems 
% \item [{RESTART}] step number from which to restart; 0 for no restart 
% \item [{NUMFIELD}] problem can consist of up to three (physical) fields 
% \item [{GRADERW}] flag to use gradient enhanced material model (for wall
% elements only) 
% \item [{MULTISC\_STRUCT}] flag to use structural multiscale analysis (for
% wall elements and static calculations only?) 
% \item [{TRACE}] secure or fast flag to switch tracing of subroutine structures
% on or of (works for debug only) 
% \end{lyxlist}
% remarks: 

% \begin{itemize}
% \item If a step for restart is specified, restart has to be given as additional
% program argument also . 
% \item The TRACE option is the first thing to be read from the input file
% and this is done in inptrace() in input\_full/input\_ctr\_head.c . 
% \item The PROBLEM TYP block must not be omitted
% \end{itemize}
% ---DISCRETISATION 

% This block gives the number of discretisations per field. Only the
% first discretisation is read from the input file. All additional ones
% have to be created according to the special needs of the algorithm
% used. Nodes and elements of additional discretisations do not occur
% within the total node and element numbers within the PROBLEM SIZE
% block. 

% read in: inpdis() in input\_full/input\_mesh.c 

% used for: indicating further discretisations of single fields 

% written by: hand 

% contains: 

% \begin{lyxlist}{00.00.0000}
% \item [{NUMFLUIDDIS}] number of discretisations in fluid field 
% \item [{NUMSTRUCDIS}] number of discretisations in structure field 
% \item [{NUMALEDIS}] number of discretisations in ale field (=1) 
% \end{lyxlist}
% remarks: 

% \begin{itemize}
% \item Second discretisations are needed for example for the projection method
% or multiscale methods. 
% \item The discretisation block may be omitted. One discretisation per field
% is assumed in this case.
% \end{itemize}
% ---IO 

% This block has an uncertain future. Its description is postponed since. 

% read in: inp{*} in input\_full/{*}.c 

% used for: something 

% written by: hand 

% contains: 

% \begin{lyxlist}{00.00.0000}
% \item [{KEYWORD}] meaning 
% \end{lyxlist}
% remarks: 

% \begin{itemize}
% \item Add a remark!
% \end{itemize}
% ---DESIGN DESCRIPTION

% This block contains the number of design entities. The design elements
% coincide with the geometrical objects created by the preprocessing
% tool. Design entities differ from FE entities and are linked within
% the program later on. Conditions are assigned to the design elements
% first. 

% read in: inp\_designsize() in input\_full/input\_design.c 

% used for: allocating design data structure 

% written by: preprocessor (GiD) 

% contains: 

% \begin{lyxlist}{00.00.0000}
% \item [{NDPOINT}] number of design nodes 
% \item [{NDLINE}] number of design lines 
% \item [{NDSURF}] number of design surfaces 
% \item [{NDVOL}] number of design volumes 
% \end{lyxlist}
% remarks: 

% \begin{itemize}
% \item The DESIGN DESCRIPTION must not be omitted! 
% \item All lines are mandatory. (ie. the 2D case requires NDVOL 0)
% \end{itemize}
% ---DESIGN POINTS

% This block contains the precise description of the single design points.
% The number of design points expected is given by the resprective value
% of the DESIGN DESCRIPTION. 

% read in: inp\_dnode() in input\_full/input\_design.c 

% used for: organising geometry and data structure 

% written by: preprocessor (GiD) 

% to read within one design point: 

% \begin{itemize}
% \item the points ID which must be in order (read rather than counted) 
% \item number of conditions to this node (is this ever used?) 
% \item nodal coordinates (always three values!) 
% \end{itemize}
% remarks: 

% \begin{itemize}
% \item The layer, the point was drawn in, is possibly written by GiD but
% not read. 
% \item The reading relies on the key words POINT and END POINT. 
% \item The design points can clearly not be omitted.
% \end{itemize}
% ---DESIGN LINES

% This block contains the precise description of the single design lines.
% The number of design lines expected is given by the resprective value
% of the DESIGN DESCRIPTION. 

% read in: inp\_dline() in input\_full/input\_design.c 

% used for: organising geometry and data structure 

% written by: preprocessor (GiD) 

% possible line types: 

% \begin{lyxlist}{00.00.0000}
% \item [{STLINE}] stright line between two points 
% \item [{NURBLINE}] stright line between two points (coming with extra data) 
% \item [{ARCLINE}] line is part of a circle 
% \end{lyxlist}
% to read within one design line: 

% \begin{itemize}
% \item the type of the line 
% \item the lines ID which must be in order (read rather than counted) 
% \item number of conditions to this line (is this ever used?) No.
% \item the two end points connected by this line 
% \item 2D center for ARCLINES only (is this really read?) 
% \item radius for ARCLINES only 
% \item initial angle for ARCLINES only 
% \item end angle for ARCLINES only 
% \item the total length for ARCLINES only 
% \end{itemize}
% remarks: 

% \begin{itemize}
% \item The layer, the line was drawn in, is possibly written by GiD but not
% read. 
% \item The reading relies on the respective key words. 
% \item The design lines can clearly not be omitted.
% \end{itemize}
% ---DESIGN SURFACES 

% This block contains the description of all design surfaces. The number
% of design surfaces expected is given by the resprective value of the
% DESIGN DESCRIPTION. 

% read in: inp\_dsurface() in input\_full/input\_design.c 

% used for: organising geometry and data structure 

% written by: proprocessor (GiD) 

% to read within one design surface: 

% \begin{itemize}
% \item the surface ID which must be in order (read rather than counted) 
% \item number of conditions to this surface (is this ever used?) 
% \item the number of lines surrounding this surface 
% \item to every line there is read 

% \begin{itemize}
% \item the line ID 
% \item the line orientation 
% \end{itemize}
% \end{itemize}
% remarks: 

% \begin{itemize}
% \item The layer, the surface was drawn in, is possibly written by GiD but
% not read. 
% \item The reading relies on the respective key words. 
% \item The design surfaces can clearly not be omitted.
% \end{itemize}
% ---DESIGN VOLUMES 

% This block contains the description of all design volumes. The number
% of design volumes expected is given by the resprective value of the
% DESIGN DESCRIPTION. 

% read in: inp\_dvolume() in input\_full/input\_design.c 

% used for: organising geometry and data structure 

% written by: proprocessor (GiD) 

% to read within one design volume: 

% \begin{itemize}
% \item the volume ID which must be in order (read rather than counted) 
% \item number of conditions to this volume (is this ever used?) 
% \item the number of surfaces surrounding this volume 
% \item to every surface there is read 

% \begin{itemize}
% \item the surface ID 
% \item the surface orientation 
% \end{itemize}
% \end{itemize}
% remarks: 

% \begin{itemize}
% \item The layer, the volume was drawn in, is possibly written by GiD but
% not read. 
% \item The reading relies on the respective key words. 
% \item The DESIGN VOLUME block can not be omitted but will be empty for 2D
% problems.
% \end{itemize}
% ---DESIGN POINT DIRICH CONDITIONS 

% This block ... 

% read in: inp{*} in input\_full/{*}.c

% used for: something 

% written by: hand contains: 

% \begin{lyxlist}{00.00.0000}
% \item [{KEYWORD}] meaning 
% \end{lyxlist}
% remarks: 

% \begin{itemize}
% \item Add a remark here!
% \end{itemize}


