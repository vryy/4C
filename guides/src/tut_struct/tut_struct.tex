%%
%%############################
\chapter{Tutorial --- A plate is modelled with Gid and solved by
  \baci{}}
\label{tut_struct:chap}

%%
%%============================
\section{Overview --- Simulation cycle with Gid and \baci{}}

Gid (\texttt{http://gid.cimne.upc.es/}) is the graphical pre- and
post-processor for \baci{}\@. If you want to solve a fresh problem
the general scheme is 

\begin{itemize}
\item creating the geometry in Gid, 
\item attributing it with material, boundary conditions, elements, etc., 
\item generating the mesh, 
\item extracting the \baci{} input file, 
\item processing the input file through \baci{}, 
\item admiring the results with Gid again. 
\end{itemize}


%%
%%============================
\section{Preprocessing}\label{tut_struct:sec:preprocessing}


\subsection{How to start Gid}

Check the Beginner's Guide at Section~\ref{beginner:sec:gid}.


\subsection{Generating geometry (plate example)}

The Gid help provides a decent tutorial on how to generate a geometry.
You can access it via \emph{help $\rightarrow$ tutorials}.

Here, a simple example is considered in which a plate is generated.
The listed method is not most quickest. The listed menu chaines can
often be cut shortly by using icons of keyboard short-cuts. 

\begin{enumerate}
\item \emph{Files $\rightarrow$ Save as $\rightarrow$} \texttt{\emph{plate.gid}}
\item Draw a rectangle \emph{Geometry $\rightarrow$ Create $\rightarrow$
Line}; \emph{Command} line: \texttt{10,-10}; \emph{Command} \texttt{10,10};
\emph{Command} \texttt{-10,10}; \emph{Command} \texttt{-10,-10}; \emph{View
$\rightarrow$ Zoom $\rightarrow$ Frame} shows area of interest;
move $+$ pointer to first point and click; choose \emph{join}; hit
\texttt{ESC}
\item Make a surface based on these lines: \emph{Geometry $\rightarrow$
Create $\rightarrow$ NURBS surface $\rightarrow$ By contour}; select
the four lines; hit \texttt{ESC}; a magenta quadrilateral appears
indicating the surface 
\item Extrude the surface: \emph{Utilities $\rightarrow$ Copy...}; \emph{Entities
type:} \texttt{\emph{Surfaces}}; \emph{Translation:} \texttt{\emph{Translation}};
\emph{First point z:} \texttt{0.0}; \emph{Second point z:} \texttt{1.0};
\emph{Do extrude:} \texttt{\emph{Surfaces}}; \emph{Select} and select
previously created surface; \emph{Finish} or \texttt{ESC}; Five new
surfaces were added. 
\item Establish the hexahedron volume: \emph{Geometry $\rightarrow$ Create
$\rightarrow$ Volume $\rightarrow$ By contour}; select the six area
surfaces; hit \texttt{ESC} twice; a blue volume is indicated 
\end{enumerate}


\subsection{Show geometry labels (node-number, line-number, etc.)}
\label{tut_struct:sec:show-geo-label}

\begin{itemize}
\item \emph{View $\rightarrow$ Label $\rightarrow$ All} shows all geometry
labels/numbers 
\item \emph{View $\rightarrow$ Label $\rightarrow$ All in $\rightarrow$
node,line,...} shows selected geometry label 
\end{itemize}

\subsection{Using \emph{lnm\_general} (static structure example)}

\begin{enumerate}
\item Choose \emph{Date $\rightarrow$ Problemtypes $\rightarrow$ lnm\_general};
give \texttt{OK}
\item Apply boundary conditions (BCs): \emph{Data $\rightarrow$ Conditions
$\rightarrow$ SingleLayer}
\item Simple support along lower edges along $x$-direction: In \emph{SingleLayer}
switch to \emph{line} (icon) and \texttt{\emph{L Dirich}}; tick \emph{1
on/off}, \emph{2 on/off} and \emph{3 on/off} to fixate deformation
to zero in $x$-, $y$- and $z$-direction; \emph{Assign}; Select
on of the lower edges in $x$-direction; Untick \emph{2} and \emph{3};
and \emph{Assign}; by selection the other lower edge in $x$-direction. 
\item Highlight the supported edges: \emph{Draw $\rightarrow$ colors};
\emph{Finish}
\item Apply load: In \emph{SingleLayer} switch to \emph{surface} (icon)
and \texttt{S Neum}; tick \emph{3 on/off} to have a load in $z$-direction;
set \emph{value3:} to \texttt{-25.0}; \emph{Assign} the surface on
which the load acts; Select top plane; \emph{Finish} or \texttt{ESC}. 
\item Highlight the load surface: \emph{Draw $\rightarrow$ colors}; \emph{Finish}
\item Setup material: \emph{Data $\rightarrow$ materials}; Select \texttt{\emph{MAT
Struct StVenantKirchhoff}}; insert for \emph{Youngs modulus} \texttt{10000};
and for \emph{Poisson ratio} \texttt{0.3}; \emph{Assign $\rightarrow$
Volume}; Select the hexahedral volume; \emph{Finish} or \texttt{ESC}. 
\item General problem data: \emph{Data $\rightarrow$ Problem Data $\rightarrow$
General Problem Data}; select tab \emph{General}; type in a \emph{Problem
title} \texttt{Plate}; \emph{Problem Type} \texttt{\emph{Structure}};
Optimize maximal number of DOFs \emph{Max NumDOF} to \texttt{3}; \emph{Static/Dynamic}
to \texttt{\emph{Static}}. 
\item Include output: In \emph{General problem data} select tab \emph{Structural
IO}; tick \emph{Struct Displ to File}; etc. 
\item More settings for static case in \emph{Data $\rightarrow$ Problem
Data $\rightarrow$ Static}. 
\item Make Geometric elements also Design elements: \emph{Data $\rightarrow$
Conditions $\rightarrow$ SingleLayer}; tick \emph{points} (icon);
select \texttt{\emph{Design Node Number}}; \emph{Assign} to all geometric
lines; \texttt{ESC}; tick \emph{lines} (icon); select \texttt{\emph{Design
Line Number}}; \emph{Assign} to all geometric lines; \texttt{ESC};
tick \emph{surfaces} (icon); select \texttt{\emph{Design Surface Number}};
\emph{Assign} to all 6 geometric surface ; \texttt{ESC}; tick \emph{volumes}
(icon); select \texttt{\emph{Design Volume Number}}; \emph{Assign}
to all 1 geometric volume; \texttt{ESC}; \emph{Close}. 
\item Assign finite element: \emph{Data $\rightarrow$ Conditions $\rightarrow$
SingleLayer}; Tick \emph{Volumes} (icon); Select \texttt{\emph{Brick}};
Keep Gauss-Points; change \emph{BrickStressIO} to \texttt{\emph{NDXYZ}},
which means the stress is noted at the geometric nodes, so Gid can
post-process the stresses. 
\item Prepare a structured mesh: \emph{Meshing $\rightarrow$ Structured
$\rightarrow$ Volumes}; Select blue plate volume; \texttt{ESC}; dialogue
\emph{Enter number of cells to assign to lines} enter \texttt{20}
along the long edges; give \emph{OK}; select the long edges (opposite
lines get selected automatically); \texttt{ESC}; dialogue Take \texttt{20}
elements along the long edges; Select them; \emph{Enter number of
cells to assign to lines} enter \texttt{5} along the short edge (thickness
direction); \emph{OK}; select one of the lines (remaing 3 get automatically
selected); \texttt{ESC}; \emph{Cancel}. 
\item Generate the mesh: \emph{Meshing $\rightarrow$ Generate mesh...};
Ignore proposed element size and give \emph{OK}; see generated mesh;
you can toggle between mesh and geometry by going to \emph{Meshing
$\rightarrow$ Mesh view}. 
\item Finally create \baci{} input file: \emph{Calculate $\rightarrow$ calculate};
this should produce \texttt{\emph{plate.dat}}. 
\end{enumerate}

\subsection{Show boundary conditions}\label{tut_struct:sec:show-bound-cond}

\begin{itemize}
\item \emph{Data $\rightarrow$ Conditions $\rightarrow$ Single Layer $\rightarrow$
Draw $\rightarrow$ Colors} shows active BCs; change between nodal,
line, facial BCs with buttons in first line of \emph{Single layer}
\end{itemize}

\subsection{Deviation of \emph{curves} from \emph{functions}}

There is a difference between the \baci{} input terms \emph{functions}
and \emph{curves}.

The latter are used as load curves, which are possibly dependent on
time. These can be assigned to the von Neumann BCs in \emph{Gid :
Data $\rightarrow$ Conditions $\rightarrow$ SingleLayer}; select
\emph{P Neum}, \emph{L Nuem}, \ldots{}as appropriate; choose number
from \emph{Load Curve} pop-down menu. The load curve is defined by
clicking on the {}``Notepad'' icon on top right corner of the \emph{SingleLayer}
window; there go to \emph{Interval Data}; define desired load curve.

The \emph{functions} are defined (for instance) in \emph{Data $\rightarrow$
Problem Data $\rightarrow$ Functions}.


%%
%%============================
\section{Processing}

More on executing the input file \texttt{\emph{plate.dat}} look at
Sec~\ref{beginner:sec:running-examples}. 


%%
%%============================
\section{Postprocessing}\label{tut_struct:sec:postprocessing}


\subsection{Show mesh}

\label{sec:show-grid}

\begin{itemize}
\item \emph{Files $\rightarrow$ Postprocess}
\item \emph{Files $\rightarrow$ open $\rightarrow$ file.flavia.res}
\item \emph{Windows $\rightarrow$ Deform mesh} or \emph{Ctrl-d}
\item select arbitrary time step 
\item \emph{Windows $\rightarrow$ View Style $\rightarrow$ Style $\rightarrow$
Body lines}
\end{itemize}

\subsection{Show results of primary unknowns}\label{tut_struct:sec:show-results}

\begin{itemize}
\item \emph{Files $\rightarrow$ Postprocess}
\item \emph{Files $\rightarrow$ open $\rightarrow$ file.flavia.res}
\item \emph{View Results $\rightarrow$ ...}
\end{itemize}

\subsection{Show stresses}\label{tut_struct:sec:show-stresses}

This feature is not implemented for all elements. Whether it is implemented
or not can be seen e.g. in the file src/\emph{element}/{*}\_inp\_ele.
There STRESS must be read in. For brick elements: 

\begin{itemize}
\item in the input file in the section STRUCTURE ELEMENTS at the end of
each element the key words STRESSES NDXYZ must be present; meaning:
stresses in the nodal points are displayed; additionally, the section
IO must be adjusted 
\item in gid the results are read in as usual (cf. \ref{tut_struct:sec:show-results}) 
\item \emph{View Results $\rightarrow$ pcarat}
\item view the stresses e.g. \emph{View Results $\rightarrow$ Contour Fill
$\rightarrow$ ...}
\end{itemize}


